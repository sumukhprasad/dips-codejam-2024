\documentclass[12pt]{report}
\usepackage[a4paper, total={7.3in, 9.7in}]{geometry}
\usepackage{amsmath}
\usepackage{upquote}
\usepackage{listings}
\usepackage{xcolor}
\usepackage{titlesec}
\usepackage{amssymb}

\definecolor{backgroundcolor}{rgb}{1, 1, 1}
\definecolor{commentstyle}{rgb}{0.365, 0.422, 0.475}
\definecolor{keywordstyle}{rgb}{0.6, 0.14, 0.576}
\definecolor{numberstyle}{rgb}{0.5, 0.5, 0.5}
\definecolor{stringstyle}{rgb}{0.77, 0.1, 0.08}

\lstdefinestyle{xcodecolor}{
    backgroundcolor=\color{backgroundcolor},   
    commentstyle=\color{commentstyle},
    keywordstyle=\color{keywordstyle},
    numberstyle=\scriptsize\color{numberstyle},
    stringstyle=\color{stringstyle},
    basicstyle=\ttfamily\footnotesize,
    breakatwhitespace=false,         
    breaklines=true,                 
    captionpos=b,                    
    keepspaces=true,                   
    numbersep=5pt,                  
    showspaces=false,                
    showstringspaces=false,
    showtabs=false,                  
    tabsize=2
}

\lstset{style=xcodecolor}

\usepackage[T1]{fontenc}
\usepackage{cascadia-code}

% Raised Rule Command:
%  Arg 1 (Optional) - How high to raise the rule
%  Arg 2            - Thickness of the rule
\newcommand{\raisedrule}[2][0em]{\leaders\hbox{\rule[#1]{1pt}{#2}}\hfill}

\setlength{\parindent}{0pt}
\titleformat{\section}
{\normalfont\Large\bfseries}{\thesection}{1em}{}[{\titlerule[0.8pt]}]

\begin{document}

	{\Large
	\textbf{Slacking Off II}}
	
	\vspace{0.4cm}
	DiPS CodeJam 24\raisedrule[0.25em]{1pt}
	\\
	% document

	\section*{Prompt}
	Bobby, having having figured out how many programs he can compile, sits down and starts compiling. Bobby now sees a problem -- in his quest for longer compile times, he's made a small error (I'll let you figure this one out yourself): 
	
	\texttt{class X: Y {}}\\
	\texttt{class Y: X {}}
	
	So he makes another list -- this time denoting which classes are subclasses of which other classes. Can you find out if such a `transitive equality' of \textbf{two} classes is present in the code?


	Hint: this can also be expressed as a graph theory problem -- \emph{Given an undirected graph and a vertex in the graph, find the number of vertices in its connected component.}
	
	\subsection*{Input Format}
	\begin{itemize}
		\item The first line of the input contains an integer $n$, denoting the number of lists.
		\item The next $n$ lines of the input each contain a space separated list of values conforming to $X=Y$, where X and Y denote class names.
	\end{itemize}
	\subsection*{Output Format}
	The first and only line of your output must contain a single integer $m$, denoting the number of lists where a transitive equality exists.
	\subsection*{Constraints}
	\begin{itemize}
		\item $ 10^2 \le n \le 10^3 $
		\item The size of each list varies between $10^2$ and $10^3$ elements.
	\end{itemize}

\section*{Sample Program}
	\lstinputlisting[language=Python]{sampleSolution.py}
	

\end{document}